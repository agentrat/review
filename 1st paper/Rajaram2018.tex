\documentclass{article}
\begin{document}
\title{Weak excitation enhances timing of post-inhibitory rebound responses}
\maketitle
\author{Ezhilarasan Rajaram, Matthew J. Fischl, Leander Mrowka, Olga Alexandrova, Benedikt Grothe and Conny Kopp-Scheinpflug}

\date{Division of Neurobiology, Department Biology II, Ludwig-Maximilians-University Munich, Großhaderner Straße 2, 82152, Planegg-Martinsried. Germany}

\section{Introduction}
Acoustic pattern recognition and vocal communication strongly depend on the temporally precise detection of onsets and offsets of sound, which are suggested to be encoded by two dissociable channels in the auditory pathway (Anderson and Linden, 2016). Neurons in the superior paraolivary nucleus (SPN) in the mammalian brainstem are a reliable early source of acoustically evoked offset responses (Kulesza et al., 2003; Kadner et al., 2006; Felix et al., 2011; Kopp-Scheinpflug et al., 2011). These offsets are generated via an inhibitory rebound mechanism that is initiated by strong glycinergic inputs and aided by the joint activation of IH currents and T-type calcium currents (Felix et al., 2011; Kopp-Scheinpflug et al., 2011). A prominent inhibitory projection onto SPN can be evoked by stimulating the medial nucleus of the trapezoid body (MNTB), generating large, fast, glycinergic inhibitory postsynaptic currents (IPSCs) which reverse at voltages well below the neurons’ resting membrane potential (Lohrke et al., 2005; Kopp-Scheinpflug et al., 2011; Jalabi et al., 2013; Yassin et al., 2014). Despite this major inhibitory input, studies of SPN responses in vivo also observed excitatory responses (Behrend et al., 2002; Dehmel et al., 2002; Kulesza et al., 2003). However, these excitatory inputs seem to be masked by strong inhibition (Kulesza et al., 2007) and though SPN neurons show some excitatory response to periodic sound stimuli, their ability to encode sinusoidal amplitude modulated sounds is rather low (Kadner and Berrebi, 2008), raising the question of what function these excitatory inputs perform. The broad frequency tuning of the contralateral excitatory input (Dehmel et al., 2002) and anatomical tracing experiments suggest octopus cells of the PVCN as one possible source of excitation in the SPN (Thompson and Thompson, 1991; Schofield, 1995; Saldana et al., 2009). After crossing the midline, thick octopus cell axons pass through the ventral and the intermediate acoustic stria to innervate SPN neurons (Kuwabara et al., 1991; Schofield, 1995). SPN neurons are also the target of decending projections from the ventral and dorsal nucleus of the lateral lemniscus [cat: (Whitley and Henkel, 1984); rat: (Bajo et al., 1993)], the tectal longitudinal column (Vinuela et al., 2011) and the medial geniculate body (Schofield et al., 2014). In addition, axon collaterals of medial superior olive neurons terminate on SPN neurons providing a local source of excitation (Kuwabara and Zook, 1999). In support of these excitatory inputs a recent study reports the presence of AMPA receptor mediated responses in the mouse SPN (Felix and Magnusson, 2016).  The information about the presence of excitatory inputs is diverse and their function for processing auditory information is not understood. Here we use a combination of immunohistochemistry, in vivo electrophysiology and patch-clamp recordings to shed light on the interaction of excitation and inhibition towards SPN function. At all postnatal ages tested, inhibitory inputs are dominant over excitatory inputs. Co-activation of such weak excitation with strong inhibition shortens the latency of the inhibitory rebound responses during electric stimulation in brain slices as well as during acoustic stimulation in vivo.

\section{Methods}
All experimental procedures were reviewed and approved by the Bavarian district government (TVV AZ: 55.2-1-54-2532-38-13) and were done according to the European Communities Council Directive (2010/63/EU). CBA/Ca mice were housed in a vivarium with a normal light dark cycle (12 hours light/ 12 hours dark). Mice of both sexes, aged P9–65, were used for the physiological and anatomical experiments (in vitro patch-clamp, single-unit in vivo recording, viral vector tracing, immunohistochemistry, and confocal microscopy).
\subsection{In vitro electrophysiology}
Mice of either sex P9-22 (n=36) were anaesthetized with isoflurane and killed by decapitation. Coronal brainstem sections (200μm-thick) containing the SOC were cut in an ice-cold high-sucrose, low-sodium artificial cerebral spinal fluid (ACSF). Brainstem slices were maintained after slicing in normal ACSF at 37°C for 30–45min, after which they were stored in a slice-maintenance chamber at room temperature (~22°C). Composition of the normal ACSF in mM: NaCl 125, KCl 2.5, NaHCO3 26, glucose 10, NaH2PO4 1.25, sodium pyruvate 2, myo-inositol 3,CaCl2 2, MgCl2 1, and ascorbic acid 0.5 pH was 7.4, bubbled with 95\% O2, 5\% CO2. For the low-sodium ACSF CaCl2 and MgCl2 concentrations were 0.1 and 4 mM, respectively, and NaCl was replaced by 200 mM sucrose. Experiments were conducted at 36± 1°C, maintained by an inline feedback temperature controller and heated stage (TC344B, Warner Instruments, Hamden, CT, USA) with the recording chamber being continuously perfused with ACSF at a rate of 1–2 ml min−1. Whole-cell patch-clamp recordings were made fromvisually identified MNTB neurons (OlympusBX51WI microscope) using an EPC10/2HEKA amplifier, sampling at 50 kHz and filtering between 2.9 and 10 kHz. Patch pipettes were pulled from borosilicate glass capillaries (GC150F-7.5, OD: 1.5mm; Harvard Apparatus, Edenbridge, UK) using a DMZ Universal puller (Zeitz), filled with a patch solution containing (in mM): K-gluconate 126, KCl 4, HEPES 40, EGTA 5 MgCl2 1 Na2phosphocreatine 5, 0.2\% biocytin, 292mOsm,(all chemicals from Sigma-Aldrich). pH was adjusted to 7.2 with KOH. Electrode resistance was between 2.4 and 6 MΩ. Synaptic potentials were evoked by afferent fiber stimulation with a bipolar electrode (FHC; Bowdoin ME, USA) placed at …….. Voltage pulses were generated by the HEKA amplifier and post-amplified by a linear stimulus isolator (PulseStimulator AM-2100). 
\subsection{In vivo physiology}
Animals were anesthetized with a subcutaneous injection of 0.01ml/g MMF (0.5mg/kg body weight Medetomidine, 5.0mg/kg body weight Midazolam and 0.05mg/kg body weight Fentanyl) and were placed on a temperature controlled heating pad (ATC1000, WPI) in a soundproof chamber (Industrial Acoustics). Depth of anesthesia was measured using the toe pinch reflex and animals responding were given supplemental MMF at 1/3 the initial dose. The mice were then stabilized in a custom stereotaxic device. An incision was made at the top of the skull and a head post was fixed to the skull using dental cement. A craniotomy was performed above the cerebellum to access the auditory brainstem. A ground electrode was placed in the muscle at the base of the neck. Glass microelectrodes were pulled from glass capillary tubes (GC150F-7.5, Harvard Apparatus, Edenbridge, UK) so that the resistance was 5-20 M when filled with 3M KCl solution or 2M potassium acetate with 2.5\% biocytin. Signals were amplified (Neuroprobe Amplifier Model 1600, A-M Systems), filtered (300-3000Hz; TDT PC1) and recorded (~50 kHz sampling rate) with an RZ6 processor (TDT). SPIKE software (Brandon Warren, V.M. Bloedel Hearing Research Center, University of Washington) was used to calibrate the speakers (MF1, Tucker Davis Technologies), generate stimuli and record action potentials. Stimuli consisted of pure tones (50-100ms duration, 5ms rise/fall time) at varying intensity (0-90dB SPL) and were presented through hollow ear bars connected to the speakers with Tygon tubing. Spike sorting and data analysis was performed offline using custom Matlab programs. At the end of the experiment, biocytin (2.5\%) was deposited at the final penetration using the current injection mode of the amplifier (A-M Systems, Neuroprobe Model 1600; +0.5 µA, 1-2min.) Thirty minutes were allowed for cellular uptake before the animal was perfused and the tissue was processed for biocytin fluorescence as described above. Recording sites were determined using the biocytin deposition as a reference for stereotaxic reconstruction. 
\subsection{Immunohistochemistry}
Sagittal brainstem sections including trapezoid body of 80µm thickness were taken from within 240µm of the midline using a vibrating microtome (Leica, VT1200s). After 3 x 10 minute washes in PBS, sections were transferred to a blocking solution containing 1\% bovine serum albumin, 1\% Triton X100, and 0.1\% saponin in PBS. Tissue was incubated for 48 hours at 4°C with the following primary antibodies diluted 1-100 in blocking solution: ……………. Tissue was then washed 3 x 10 minutes in PBS at room temperature, before incubation for 24 hours at 4°C with secondary antibodies diluted 1:200 in blocking solution: Cy3 donkey anti-mouse, Alexa-488 donkey anti-rat (715-545-153 and 715-166-151 respectively, Dianova, Hamburg, DE) sections were rinsed 3 x 10 min in PBS, and coverslipped with Vectashield mounting medium.
\subsection{Confocal microscopy}
Confocal optical sections were acquired with a Leica TCS SP5-2 confocal laser-scanning microscope (Leica Microsystems, Mannheim, Germany) equipped with HCX PL APO CS 20X / NA0.7 and HCX PL APO Lambda Blue 63x / NA1.4 immersion oil objectives. For each optical section the images were collected sequentially for two fluorochromes. Stacks of 8-bit grayscale images were obtained with axial distances of 290 nm between optical sections and pixel sizes of 120-1520 nm depending on the selected zoom factor and objective. To improve the signal-to-noise ratio, images were averaged from three successive scans. 
\subsection{Experimental design and statistical analysis}
Data are presented as mean ± s.e.m. with p values, degrees of freedom (df), and sample size (n). Statistical analyses of the data were performed with SigmaStat/SigmaPlot™ (SPSS Science, Chicago, IL). Comparisons between different data sets were made depending on the distribution of the data using parametric tests for normally distributed data (two-tailed Student’s t-test for comparing two groups and ANOVA for comparing three or more groups). When the normality assumption has been violated, non-parametric tests (Mann-Whitney Rank Sum Test for comparing two groups and ANOVA on ranks for comparing three or more groups) were used. Normality was tested by the Shapiro-Wilk Test. Differences were considered statistically significant at p<0.05. Intrinsic properties as well as PSC amplitudes and kinetics were analyzed using Stimfit software (Guzman et al., 2014). For data acquired with patch-clamp recording (figures 1, 2) or in vivo single unit recording (figure 8); n is the number of neurons, with one cell per brain slice (to enable anatomical reconstruction of each individual neuron), 2-3 brain slices per animal and at least 3 animals per group. Neurons’ position was determined after recording using ImageJ, measuring the distance between the pipette tip and the midline under a 4x lens, from a tiff file taken with a Pike F145B CCD. Test details are indicated in each of the respective results sections.

RESULTS
Excitatory inputs shorten latency of post-inhibitory rebound responses.
The majority of mouse SPN neurons (15/17) show postinhibitory rebound firing in response to contralateral sound stimulation in vivo, marking the end of the sound stimulus (Fig. 1A, B). Only 2/17 SPN neurons did not exhibit postinhibitory rebound firing, but rather sole excitatory responses with a phasic-tonic temporal response (Fig. 1C). Closer inspection of SPN neurons with offset-type responses revealed that about half of them showed some excitatory responses in addition to the inhibition (Fig. 1B). A function of offset latencies vs. sound intensity visualizes firstly that the offset latencies are constant over a very large range of sound intensities and secondly that the offset responses in the on-off type of SPN neurons are faster (4.86 ±0.86ms; n=6) compared to the offset only responses (13.83 ±3.09ms; n=9; two-tailed t-test: P=0.0386; Fig. 1D). Besides shorter latencies, SPN neurons with on-off responses also had higher spontaneous firing rates (18.78 ±3.21Hz; n=6) compared to neurons with offset-only responses (5.56 ±2.62Hz; n=9; two-tailed t-test: P=0.007). Variability of the offset latency (jitteron-off: 0.78 ±1.27ms; n=6; jitteroffset-only: 0.98 ±3.36ms; n=9; Mann-Whitney Rank Sum Test: P=0.316) of the neurons of either response type was not significantly from each other. 

SPN neurons are innervated by glycinergic and glutamatergic synapses.
In vivo recordings revealed that excitation as well as inhibition plays a role during SPN signal processing with inhibition being more prominent. This is corroborated by immunocytochemistry revealing strong expression of the glycine transporter type 2 which labels glycinergic synaptic terminals at SPN neurons of prehearing and posthearing mice (Fig. 2). The presence of excitatory synapses was confirmed by labeling for the vesicular glutamate transporter (VGLUT) types 1-3. Although no quantification of the immunocytochemical results was attempted, VGLUT2 expression seemed stronger than VGLUT1 expression while VGLUT3 expression was quite low at both ages tested (Fig. 2G-L). GlyT2 was strongly expressed around the soma of the SPN neurons suggesting that the inhibitory inputs innervate mainly the somata (Fig. 2G-L). VGLUT1 labeling was sparse but like inhibition close to the soma of the neurons (Fig. 2Hi, ii). In contrast, VGLUT2 seems to be expressed mainly on the dendrites (Fig. 2Ji, ii).

NMDA receptors are expressed in SPN neurons of pre- and posthearing mice.
In the auditory brainstem, glutamate released from excitatory synaptic inputs typically activates AMPA or NMDA receptors. SPN neurons have previously been shown to express AMPA receptors containing the GluR2 subunit rendering the AMPA receptors calcium impermeable (Felix and Magnusson, 2016). Here we used immunocytochemistry and patch-clamp recording to test whether NMDA receptors are expressed and can serve as an alternative entry point for calcium into SPN neurons. Expression of NMDA receptors in mature SPN neurons was present, although weaker compared to neurons in the MNTB or LSO (Fig. 3A-D).  Electrophysiological NMDA currents were sensitive to D-AP5 and showed the characteristic voltage-dependence causing larger currents once the membrane voltage reaches depolarized values (Fig. 3 E, F). The NMDA currents in SPN neurons were quite small in prehearing mice (P9-11: 162.84 ±23.34pA; n=11) with values significantly decreasing with postnatal development (P12-14: 66.02 ±14.00; n=7; P15-22: 59.84 ±12.58; n=8; Bonferroni test; P≤0.05; Fig. 3G). Decay time constants decreased with postnatal development reaching values of 7.16 ±1.10ms (n=7) in the third postnatal week (Fig. 3H).  Due to their small amplitude and slow time constant NMDA responses might be involved in intracellular second messenger signaling rather than electrical voltage signaling. In the following experiments AMPA and NMDA responses are blocked by a cocktail of DNQX and D-AP5 and are together contrasted against glycinergic inhibitory currents. 

Inhibition dominates over excitation in SPN neurons.
Glutamatergic excitatory postsynaptic currents (EPSCs) were regularly activated in SPN neurons when electrically stimulating the intermediate acoustic stria medial to the SPN and just dorsal to the MNTB. The EPSCs were of moderate amplitudes which did not change significantly between prehearing and posthearing ages (EPSCP9-11: 379.62 ±100.73pA; n=9; EPSCP12-14: 577.57 ±156.23pA; n=15; EPSCP15-18: 396.48 ±62.90pA; n=13; EPSC>P19: 478.97 ±94.90pA; n=6; ANOVA: P=0.382; Fig. 4A, C). Glycinergic inhibitory postsynaptic currents (IPSCs) were evoked when electrically stimulating directly the unilateral MNTB and also did not change significantly across hearing onset (IPSCP9-11: 3960.34 ±966.22pA; n=11; IPSCP12-14: 2112.61 ±761.42pA; n=7; IPSCP15-18: 2324.76 ±542.56pA; n=6; IPSC>P19: 2454.45 ±634.15pA; n=6; ANOVA: P=0.668; Fig. 4B, D). However, in contrast to the EPSCs, IPSC amplitudes were large at all ages tested. This difference between the strength of excitation vs. that of inhibition was even more obvious when both were expressed as conductance (Fig. 4 E,F). The decay time constant of the EPSCs were fast (EPSC tauP9-11: 0.95 ±0.16ms; n=9; EPSC tauP12-14: 1.06 ±0.19ms; n=8; EPSC tauP15-18: 0.58 ±0.25ms; n=6; EPSC tau>P19: 0.87 ±0.32ms; n=4) and did not change significantly across hearing onset at P12 (ANOVA: P=0.211; Fig. 4G). IPSCs were still subject to a developmental decrease in time constant with the only significant drop at hearing onset (IPSC tauP9-11: 2.15 ±0.21ms; n=10; IPSC tauP12-14: 1.17 ±0.10ms; n=8; IPSC tauP15-18: 1.10 ±0.04ms; n=9; IPSC tau>P19: 1.22 ±0.13ms; n=7; ANOVA/Dunn’s: P=0.003; Fig. 4H).

Excitatory shunting shortens inhibitory rebound latency.
Excitatory inputs to SPN neurons are not a prerequisite for generating offset responses (Kopp-Scheinpflug et al., 2011). However, it is not yet known whether this excitation might somehow shape the offset responses. Here we compared synaptically evoked offset responses in SPN neurons in control condition and during blockade of excitatory inputs (Fig. 5A, B) while simultaneously stimulating both the ventral and the intermediate acoustic stria using a fork-like stimulating electrode. The reliability of generating an offset response in each of 10 consecutive input trains was 90 ±9\% in controls and did not change significantly during blockade of excitation (86 ±7\%; Mann-Whitney Rank Sum test: P=0.551; Fig. 5C). The number of action potentials within each burst of the offset response was also not significantly different between control condition (2.8 ±0.5) and the blockade of excitation (2.4 ±0.6; Mann-Whitney Rank Sum test: P=0.359; Fig. 5D). To probe whether tonically active excitation is present in the SPN, we compared the resting membrane potential before and during blockade of excitation, but no significant difference was found (Vrestcontrol: -61.8 ±2.4mV; Vrestdrug: -61.1 ±2.0mV; t-test: P=0.832; Fig. 5E). Despite no change in the resting membrane potential, the application of drugs to block glutamatergic transmission caused a change in the amplitude of the evoked postsynaptic potentials in response to a train of stimuli (Fig. 5F). The stimulation of synaptic input triggered a combined response of excitation and inhibition with a hyperpolarizing net amplitude.  However, blocking excitation caused the IPSPs to drop towards more hyperpolarizing voltages by an average of 1.7 ±0.3mV (n=5, Fig. 5F,G). Along with the drop in IPSP amplitude, blockade of excitation caused a 30.7 ±6.7\% increase in the latency of synaptically evoked offset action potentials (n=5; Fig. 5F, H). The latencies of synaptically evoked responses varied between cells from 5.6 to 68.5ms (control) and from 7.2 to 84.2ms (DNQX/AP5), but each pair showed an increase in latency during DNQX/AP5 application. The percentage increase from 100\% (control) to 130.7 ±6.7\% (DNQX/AP5) was statistically significant (n=5; paired t-test: P=0.01).
Computational modeling…
We have two cell types:	a) inhibition only (that’s what we had in the 2011 paper) and
b) inhibition and excitation (new here: cells have shorter offset latency and higher spontaneous rates)
Question for the model:	1) Can this weak excitation (that does not affect number or reliability of offset spikes) act in a shunting way to speed up the latency (similar to what IH does?)?
	2) Does higher spontaneous activity and a 2mV depolarization in the IPSP train act as “background noise” and affect thresholds of gap detection? E.g. (Wilson and Walton, 2002)?

DISCUSSION
1500 words
Background noise improves gap detection in IC neurons (Wilson and Walton, 2002)!!


LEGENDS
Figure 1. SPN neurons that receive excitation as well as inhibition seem to have shorter offset latencies in vivo. A,C,E. Raster plots from SPN neurons with three different spiking profiles. B,D,F. PSTHs from the cells in the left column at 80SPL (20 trials) reveal primary-like (A,B), onset-offset (C,D), and offset (E,F) firing patterns. G. Graph of response types in SPN shows the majority of cells exhibit offset patterns. H. Offset latency vs. spiking pattern shows that offset cells receiving excitatory input (on-off) have significantly shorter offset latencies compared to offset only cells.
Figure 2. Histochemical profile of inputs to the SPN during development. Glycinergic input is depicted by neuronal glycine transporter (GlyT2; Millipore; 1:1000) labeling and forms the most prominent input to SPN prehearing and posthearing mice. Glutamatergic inputs are shown by labelling for the vesicular glutamate transporters (VGLUT1; SySy; 1:2000; VGLUT2; SySy; 1:1000; VGLUT3; Alomone; 1:300).  While all VGLUTs are present, VGLUT2 is most prominent in both age groups.
Figure 3. SPN neurons express NMDA receptors which mediate small currents at all developmental ages tested. A) SOC overview for NMDA staining. B-D) NMDA receptors are present in a subset of SPN neurons. E,F) NMDA currents are strongly non-linear due to the Mg2+ block. G) Peak NMDA currents decline at hearing onset, but remain stable after that. H) Decay time constants accelerate between the second and third postnatal week.
Figure 4. Development of IPSCs and EPSCs  in SPN. A, B) Voltage-clamp traces of EPSCs (A) evoked by stimulating IAS and IPSCs (B) evoked by stimulating MNTB (avg of 10 traces). C, D) Average EPSC and IPSC amplitudes, respectively, to max. stimulation (100V). Over the four age groups no significant differences were found. E, F) EPSCs and IPSCs expressed as conductance reveals that IPSG is around ten times larger than EPSG. G, H) Average decay tau of EPSCs and IPSCs. TauIPSC before hearing onset (P9-11) is significantly slower than other age groups, indicating that IPSCs accelerate with age, aiding high temporal precision in SPN function. 
Figure 5. Excitation contributes to timing of SPN offset responses. Synaptically evoked IPSPs (asterisks) are followed by burst of offset action potentials in control condition (A) and after blockade of excitation by ??mM DNQX and ??mM AP5 (B). Stimulus artifacts are removed for clarity (arrows). C) Reliability of offset responses was assessed by testing each of 10 consecutive traces (as shown in A and B) for the presence of at least one rebound action potential. 100\% means that there was at least one action potential in each trace. D) The number of action potentials in each rebound burst was averaged over 10 trials. Spontaneous action potentials later in the trace were not included. E) Resting membrane potentials were averaged over 10ms before the start of synaptic stimulations. F) 100Hz synaptic stimulation elicited an offset response in control (black) and during blockade of excitation (blue). G) The difference in average amplitudes of the synaptic potentials before and after blockade of excitation was plotted against stimulus number within the train (n=5, 10 trials each). H) Increase in offset latency during blockade of inhibition in percent.
Figure 6. Model to look at the effect on gap thresholds?

REFERENCES
Anderson LA, Linden JF (2016) Mind the Gap: Two Dissociable Mechanisms of Temporal Processing in the Auditory System. The Journal of neuroscience : the official journal of the Society for Neuroscience 36:1977-1995.
Bajo VM, Merchan MA, Lopez DE, Rouiller EM (1993) Neuronal morphology and efferent projections of the dorsal nucleus of the lateral lemniscus in the rat. The Journal of comparative neurology 334:241-262.
Behrend O, Brand A, Kapfer C, Grothe B (2002) Auditory response properties in the superior paraolivary nucleus of the gerbil. Journal of neurophysiology 87:2915-2928.
Dehmel S, Kopp-Scheinpflug C, Dorrscheidt GJ, Rubsamen R (2002) Electrophysiological characterization of the superior paraolivary nucleus in the Mongolian gerbil. Hearing research 172:18-36.
Felix RA, 2nd, Magnusson AK (2016) Development of excitatory synaptic transmission to the superior paraolivary and lateral superior olivary nuclei optimizes differential decoding strategies. Neuroscience 334:1-12.
Felix RA, 2nd, Fridberger A, Leijon S, Berrebi AS, Magnusson AK (2011) Sound rhythms are encoded by postinhibitory rebound spiking in the superior paraolivary nucleus. The Journal of neuroscience : the official journal of the Society for Neuroscience 31:12566-12578.
Guzman SJ, Schlogl A, Schmidt-Hieber C (2014) Stimfit: quantifying electrophysiological data with Python. Front Neuroinform 8:16.
Jalabi W, Kopp-Scheinpflug C, Allen PD, Schiavon E, DiGiacomo RR, Forsythe ID, Maricich SM (2013) Sound localization ability and glycinergic innervation of the superior olivary complex persist after genetic deletion of the medial nucleus of the trapezoid body. The Journal of neuroscience : the official journal of the Society for Neuroscience 33:15044-15049.
Kadner A, Berrebi AS (2008) Encoding of temporal features of auditory stimuli in the medial nucleus of the trapezoid body and superior paraolivary nucleus of the rat. Neuroscience 151:868-887.
Kadner A, Kulesza RJ, Jr., Berrebi AS (2006) Neurons in the medial nucleus of the trapezoid body and superior paraolivary nucleus of the rat may play a role in sound duration coding. Journal of neurophysiology 95:1499-1508.
Kopp-Scheinpflug C, Tozer AJ, Robinson SW, Tempel BL, Hennig MH, Forsythe ID (2011) The sound of silence: ionic mechanisms encoding sound termination. Neuron 71:911-925.
Kulesza RJ, Jr., Spirou GA, Berrebi AS (2003) Physiological response properties of neurons in the superior paraolivary nucleus of the rat. Journal of neurophysiology 89:2299-2312.
Kulesza RJ, Jr., Kadner A, Berrebi AS (2007) Distinct roles for glycine and GABA in shaping the response properties of neurons in the superior paraolivary nucleus of the rat. Journal of neurophysiology 97:1610-1620.
Kuwabara N, Zook JM (1999) Local collateral projections from the medial superior olive to the superior paraolivary nucleus in the gerbil. Brain research 846:59-71.
Kuwabara N, DiCaprio RA, Zook JM (1991) Afferents to the medial nucleus of the trapezoid body and their collateral projections. The Journal of comparative neurology 314:684-706.
Lohrke S, Srinivasan G, Oberhofer M, Doncheva E, Friauf E (2005) Shift from depolarizing to hyperpolarizing glycine action occurs at different perinatal ages in superior olivary complex nuclei. The European journal of neuroscience 22:2708-2722.
Saldana E, Aparicio MA, Fuentes-Santamaria V, Berrebi AS (2009) Connections of the superior paraolivary nucleus of the rat: projections to the inferior colliculus. Neuroscience 163:372-387.
Schofield BR (1995) Projections from the cochlear nucleus to the superior paraolivary nucleus in guinea pigs. The Journal of comparative neurology 360:135-149.
Schofield BR, Mellott JG, Motts SD (2014) Subcollicular projections to the auditory thalamus and collateral projections to the inferior colliculus. Frontiers in neuroanatomy 8:70.
Thompson AM, Thompson GC (1991) Posteroventral cochlear nucleus projections to olivocochlear neurons. The Journal of comparative neurology 303:267-285.
Vinuela A, Aparicio MA, Berrebi AS, Saldana E (2011) Connections of the Superior Paraolivary Nucleus of the Rat: II. Reciprocal Connections with the Tectal Longitudinal Column. Frontiers in neuroanatomy 5:1.
Whitley JM, Henkel CK (1984) Topographical organization of the inferior collicular projection and other connections of the ventral nucleus of the lateral lemniscus in the cat. The Journal of comparative neurology 229:257-270.
Wilson WW, Walton JP (2002) Background noise improves gap detection in tonically inhibited inferior colliculus neurons. Journal of neurophysiology 87:240-249.
Yassin L, Radtke-Schuller S, Asraf H, Grothe B, Hershfinkel M, Forsythe ID, Kopp-Scheinpflug C (2014) Nitric oxide signaling modulates synaptic inhibition in the superior paraolivary nucleus (SPN) via cGMP-dependent suppression of KCC2. Frontiers in neural circuits 8:65.




\end{document}
