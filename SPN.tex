
\documentclass{article}
\begin{document}
\title{SPN}
\maketitle

\section{Introduction}

What is the SPN ?
Superior paraolivary nucleus (SPN) is  a prominent part of the rodent superior olivary complex, a group of nuclei in the brainstem involved in auditory processing. The potential homologue of SPN in carnivores and bats is the dorsomedial periolivary nucleus (DMPO).


Rats



Mice


Gerbils

Guinea pigs

Chinchillas


Ferrets

Cats

Bats


Moore 1988 (Ferrets) ; Osen et al. 1984; Smith et al. 1991) and bats (Grothe et al. 1994; Zook and Casseday 1982; Zook and DiCaprio 1988)

In ferrets, the projection of SPN to IC is not very strong. Retrograde labelling shows weak staining in SPN. (Moore 1988).

In bats, DMPO receives bilateral inputs from VCN. Contralateral input seems much stronger (Grothe 1994)

DMPO receives input from globular bushy cell axons (Smith et al 1991). Function of DMPO ? How is it different to SPN ?


SPN does not receive descending input from the IC.

Can phase lock to AM stimuli up to 200 Hz. 

Does not project to cochlea or cochlear nucleus as many periolivaary nuclei  do. ???

Ipsilateral DMPO in guinea pigs, does project to VCN (Shore 1991).





In chinchillas and guinea pigs, SPN has glycinergic neurons that project to IC. Radioactive 3H glycine injected in IC to trace glycinergic projections to IC (Saint Marie 1990)




\section{Papers I can't access:}

1.Harrison J.M., Feldman M.L.Anatomical aspects of the cochlear nucleus and superior olivary complex
(7th edn.)Neff W.D. (Ed.), Contributions to Sensory Physiology, Vol. 4, Academic Press, New York (1970), pp. 95-142

https://www.sciencedirect.com/science/article/pii/B9780121518042500103/pdfft?md5=f925d2650d28bc3dfbf6b8b1fe09ac77\&pid=1-s2.0-B9780121518042500103-main.pdf




\section{Methods}

Ih decay tau : from peak to 1 sec mark in the trace. delta t, between cursor 1 and 2 = 0.955 secs

\end{document}

